\documentclass[12pt]{article}
\usepackage{float}
\title{\LARGE \textbf{Lgebra: A Symbolic Language} \\\large \textbf{CS****: Compiler-II Course Project}}
\author{\textbf{Group **}}

\begin{document}

    \maketitle
    \tableofcontents

    \newpage
    \section{Introduction}
    \section{Why Lgebra?}
    \section{Language Specifications}
    \subsection{Keywords}
    \begin{table}[H]
        \centering
        \begin{tabular}{|c|c|c|}
            \hline
            \textbf{Keywords} & \textbf{Description} & \textbf{Example} \\
            \hline
            if & & \\
            \hline
            else & & \\
            \hline
            until & & \\
            \hline
            repeat & & \\
            \hline
            for & & \\
            \hline
            break & & \\
            \hline
            continue & & \\
            \hline
            return & & \\
            \hline
        \end{tabular}
    \end{table}
    \subsection{Data Types}
    \begin{table}[H]
        \centering
        \begin{tabular}{|c|c|c|}
            \hline
            \textbf{Data Types} & \textbf{Description} & \textbf{Example} \\
            \hline
            int & & \\
            \hline
            long & & \\
            \hline
            float & & \\
            \hline
            real & & \\
            \hline
            complex & & \\
            \hline
            vector$<$Data Type$>$ & & \\
            \hline
            curves & & \\
            \hline
        \end{tabular}
    \end{table}
    \subsection{Identifiers}
    \subsubsection{Rules}
    \begin{enumerate}
        \item All identifiers should start with alphabets
        \item 
    \end{enumerate}
    \subsubsection{Reserved Identifiers}
    \begin{enumerate}
        \item Keywords and Datatype are reserved Identifiers
        \item Constants like pi, e, ... are reserverd Identifiers
    \end{enumerate}
    \subsection{Declarations}
    \subsubsection{Curves}
    \begin{enumerate}
        \item Curve should be declared as follows
        \begin{verbatim}
curve curve_name(commma seperated variables) 
= Expression in terms of independent variable
        \end{verbatim}
        \item Every curve should have atleast one independent variable (like x in f(x))
        \item Apart from independent variables, other variable in expression should be declared and defined.
        \item By default the return type of function is real. Hence it need not to be mentioned.
        \item In following example, both x is different
        \begin{verbatim}
int x = 1;
curve f(x) = x^2+1;
        \end{verbatim}
    \end{enumerate}
    \subsubsection{Other Non-Curves}
    \begin{enumerate}
        \item Other declaration are C like declaration.
    \end{enumerate}
    \subsection{Expression}
    \subsubsection{Curve}
    \begin{enumerate}
        \item Curve evaluation syntax is similar to call
            \begin{itemize}
                \item Assume declaration is \textbf{curve f(x, y)}
                \item \textbf{f(a)}: Curve f is called with value of x. Is similar to f(x=x)
                \item \textbf{f(a,b)}: Curve f is called with value of x and y
                \item \textbf{f(a,b,c)}: Error. Excess number of arguments
                \item \textbf{f(x=a, y=b)}: Curve f is called with value of x as a and y as b.
                \item \textbf{f(x=a, y=b, z=c)}: Curve f is called with value of x as a, y as b and z as c. \textbf{No Error:}
                    z will be substituted be with c. If there is no z then there will be no effect of z=c;
            \end{itemize}
    \end{enumerate}
    \subsubsection{Non-Curve}
    Similar to C
    \subsection{Constants}

    \subsubsection{Built-In constants}
    \begin{table}[H]
        \centering
        \begin{tabular}{|c|c|c|}
            \hline
            \textbf{Name} & \textbf{Value} & \textbf{Description} \\
            \hline
            e&  2.721 & Euler Constant  \\
            \hline
        \end{tabular}
    \end{table}
    \subsubsection{User-defined constants}
    Explain About Long long constant, float constant , complex constant etc
    \subsection{Functions}
    \subsubsection{Built-In Functions}
    \begin{enumerate}
        \item sum
        \item trigonometric functions (return type: curve; arguments: (curve)) \begin{enumerate}
            \item sin
            \item cos
            \item tan
            \item sec
            \item cosec
            \item cot
        \end{enumerate}

    \item curve input\_poly(int n)
    \item void print\_poly(curve c)
    \item 
    \end{enumerate}
    \subsubsection{User-defined Functions}
    \begin{enumerate}
        \item User Defined Function should be defined as follows:
    \end{enumerate}
    \subsection{Structs}
    \begin{enumerate}
        \item C like functionalities
    \end{enumerate}
    \subsection{Vectors}
    \begin{enumerate}
        \item Explain Operation on Vectors and how to declare it.
    \end{enumerate}
    \subsection{Error Analysis}
    \begin{enumerate}
        \item Explain try and catch block
    \end{enumerate}

    \section{Other Functionalities}
    \subsection{Operator and Function Overloading}
    \subsection{Irrational Mathematics}


    \section{Compilation Steps}
    \section{Performance Analysis}
    \section{Drawbacks}
    \section{Future Scope}
    \section{Conclusion}

\end{document}